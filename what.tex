\chapter{What?}

If you're getting into a modern combat flight sim like BMS or DCS,
there's lots of great guides to learn how to fly:
\href{https://www.mudspike.com/chucks-guides/}{Chuck's},
\href{https://www.youtube.com/channel/UCvXXUrGCF3wV3bbZ6pFQ00g}{Jabbers},
\href{https://www.youtube.com/user/RedKiteRender}{RedKite},
and
\href{https://www.youtube.com/channel/UCqH078Ef0HENo01LF3xwIvA}{Crash Laobi},
just to name a few.
They'll show you how to start up, take off, fly around
without ripping your wings off, shoot some weapons, and land.

But once you've got a decent grasp on how to fly a jet,
guides for \emph{employing} that jet seem harder to come by.
Once you know how not to crash the thing, you probably want to learn how to:
\begin{itemize}
\item Dogfight
\item Shoot other planes beyond visual range \ac{(bvr)}
\item Defend yourself from a surface-to-air missile \ac{(sam)}
\item Put a bomb on a target
\item Act as a wingman or a flight lead, and fly in formation.
\item Communicate with other pilots, and sound cool doing so
\end{itemize}
% \dots all without taking a missile to the face.

Specific tactics vary from era to era and jet to jet,
but many of the same fundamentals have held for the last 50
years.\punckern\footnote{We'll try to discuss how evolving technology
enables newer tactics as we go.}
This won't be a comprehensive guide,
and contributions are welcome.
But it should outline the basics, and point you to resources where you can
learn more.

\section{On Alphabet Soup}

Militaries and pilots both love acronyms,
and the jargon you get when you cross the two is an ungodly mess.
I'll try to introduce relevant acronyms and terms where I can,
but I'm bound to miss a few.
Thankfully, the Internet is full of useful aids---hopefully they can help you
out.\punckern\footnote{%
\href{https://www.fighterpilotpodcast.com/glossary/}{The Figheter Pilot Podcast's glossary}
is a great example.}
