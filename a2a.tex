\chapter{Air to Air}

Intercepting and shooting other airplanes is about managing \emph{time and space}.
If you know a bandit is 100 miles away from your jet,
you have lots of time to create, execute, and evaluate a plan.
If someone pops-up five miles away and starts shooting at you,
All you can do is react.
Situational awareness \acronym{(sa)} is king.

\section{BVR Searching}

You've got to find the bad guys before you can shoot them,
so use the resources you have to get a \introduce{picture} of the airspace.
This could include airborne or ground-based command and control
\acronym{(c2)}, datalink, your jet's own radar,
and other planes in your flight.
Communicate and share information with each other---see \chapref{brevity}.

Searching with your radar is something that comes with practice,
but understand that it's a fundamental tradeoff between searching a big area
slowly, or searching a smaller area rapidly.
You change the size of this area by controlling the horizontal azimuth
and the number of vertical bars in your scan.
Imagine searching a dark room with a narrow flashlight beam.
You can examine the whole room by swinging the flashlight in wide arcs
in front of you (azimuth), and repeating those arcs up and down (bars).
Or you can focus on a particular corner by swinging the light narrow arcs
repeated over fewer vertical motions.
There's no ``right'' approach---it depends on the situation.

\section{BVR engagements}

You've found something you want to shoot. Great!
Next make sure to \emph{positvely identify}
that it's something worth shooting with whatever you have at your disposal:
\acronym{iff} transponders,
declarations from \acronym{c2},
non-cooperative-target recognition \acronym{(nctr)}

Once you've found some contacts you're interested in,
use tools like track-while-scan \acronym{(tws)} to follow the target without
sacrificing \acronym{sa}. Once you single-target-track
