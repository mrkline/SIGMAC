\chapter{Air to Air: Beyond Visual Range}

Intercepting and shooting other airplanes is about managing \emph{time and space}.
If you know a bandit is 100 miles away from your jet,
you have lots of time to create and execute a game plan.
If someone pops up five miles away and starts shooting at you,
all you can do is react.
Situational awareness \ac{(sa)} is king---knowing what's going on around
you will keep you alive and let \emph{you} decide how a fight will unfold.

\section{Searching}

You've got to find the bad guys before you can shoot them,
so step one is to use the tools in your plane to build a \emph{picture}
of the airspace around you.
These tools include your radar,
radio communication with command \& control \ac{(c2)} assets
and other airplanes, datalink,
and your radar warning receiver \ac{(rwr)}.

Each of these sources has its own advantages and disadvantages.
For example,
your radar's view is limited to a cone off your nose,
but it provides rapid, accurate updates about the things it sees.
On the opposite side of the spectrum,
\ac{c2} often has the most complete picture of the airspace,
but slower response times and less accurate locations
since they need to convey that information by talking on the radio.

\subsection{Radar}

Let's talk about the radar first,
since it's your primary tool for finding other planes.
Searching for targets comes with practice.
While details vary from plane to plane,
all of them\footnote{Ignoring very modern active electronically scanned array
\ac{(aesa)} radars, which you probably won't find in a sim.}
work by emitting a beam of energy from an antenna in your nose.
Because this beam is fairly narrow,
the radar ``looks around'' by pointing the antenna with mechanical gimbals.
Imagine you're in a dark room with a flashlight.
You can slowly reveal the whole room by swinging the flashlight in wide arcs
in front of you (increasing your scan azimuth),
and repeating those arcs up and down, one on top of each other
(scan bars).
Or you can focus on a particular corner by swinging the light in short arcs
repeated over fewer vertical bars.

This is the fundamental tradeoff: you need to choose between
slowly searching a big area, or rapidly searching a small area.
If you're not sure where targets are,
a larger scan will be more likely to find them.
A smaller scan, meanwhile, is useful for quickly narrowing down a target's
location once you have a rough idea from other sources.
If you're running away from a target, but want to quickly turn around and
snap a shot off,
you should command a narrow scan in the direction you're turning
to quickly acquire the target.

Once you've found some contacts you're interested in,
use radar modes like \introduce{track-while-scan} \ac{(tws)}
to follow the target.
This will provide you with frequent updates on the target's position, heading,
and speed, while still providing \ac{sa} of the surrounding airspace.

\subsection{Command and Control}

Assuming you're on the right radio frequency,
\ac{c2} assets can can quickly hand you key information
about the air picture. See \textbf{Picture}, \textbf{Bogey Dope},
and \textbf{Declare} in \chapref{brevity}.

\subsection{Datalink}

Newer jets often have a \introduce{datalink} network which allows
planes to share an air picture, which you can then display in your cockpit.
While you still need to use your own radar to guide missiles,\punckern%
\footnote{Modern datalink systems actually \emph{can} guide another plane's
missiles, but, like \ac{aesa} radars, you probably won't see it in a sim.}
datalink contacts can be an excellent hint as to where you should look.
Working with your wingman to share data and quickly scan the airspace around
you is a vital skill.

\subsection{Radar Warning Receiver}

An \ac{rwr} works by detecting incoming radar pulses
with antennas spread around the plane,
then using a catalog of known emitters to categorize the threat
and show you its \emph{rough} direction and distance.\punckern\footnote{Real
fighter pilots often point out that \ac{rwr}s---especially older
ones---provide much less reliable info than sims often depict.}
Their capabilities vary wildly depending on their vintage and country of origin:
modern American \ac{rwr}s give fairly precise azimuth information
and can display dozens of emitters at once.
Russian \ac{rwr}s usually only show the highest threat or two,
but also display the emitter's signal strength and relative altitude.

\section{IFF}

You've found something you want to shoot. Great!
Next you need to get \introduce{identification, friend or foe} \ac{(iff)}.
Positive \textsc{id} can come from multiple factors, including:
\begin{itemize}
\item Visually identifying the target with a targeting pod or your eyeballs
\item Using \ac{iff} transponders, which work by sending a coded radio signal
    to the aircraft in question and expecting a certain response back.
\item Declarations from \ac{c2}
\item Non-cooperative target recognition \ac{(nctr)},
    a feature of some radars that can classify an aircraft based
    on the size and shape of its engines' fan blades.
    (This only works in \ac{stt}, at shorter ranges,
    against an enemy heading straight towards you or away from you.)
\end{itemize}
Once you're sure they're hostile, let's talk about shooting missiles.

\section{Missile kinematics and you}

At the risk of stating the obvious, you win a missile fight by making sure yours
hits them before theirs hits you.
But outside the movie \emph{Behind Enemy Lines},
missiles don't endlessly chase down their targets.
The rocket motor burns for a few seconds,
accelerating the missile to very high speeds (Mach 2+),
then the missile coasts to its target.\punckern\footnote{Many
\ac{bvr} missiles also climb---or \introduce{loft}---while the motor burns
to reach thinner air, where they can coast further.}

This makes \introduce{beyond visual range} \ac{(bvr)} engagements
a game of energy management.
You want to maximize your missile's energy so that it can get to your target.
At the same time, you want \emph{deplete} any incoming missile's energy
so that it falls out of the sky trying to catch you.
Let's talk about the ``not getting shot'' part first.

\textbf{\textsc{Do not fly directly towards someone shooting at
you.}}\punckern\footnote{Or who \emph{might} be shooting at you---we'll talk
about active radar homing \ac{(arh)} missiles shortly.}
Doing so gives an incoming missile the easiest possible job:
it can just fly a straight line, right into your dumb face.
Instead, you \introduce{crank}.
Once an enemy gets close enough that they could feasibly shoot you,
turn so that they're on the side of your radar scope,
a few degrees from its gimbal limit.
This does two important things:
first, it maximizes the distance an incoming missile has to fly to get to you
(while keeping the target on your radar).
Second, it forces an incoming missile to \emph{continuously turn} while it
flies towards you
And because missiles, like airplanes, turn by sticking their control surfaces
into the wind stream, this slows them down.
Speed is also a factor---the faster you're going,
the further a missile has to chase you down.

If you're up against a competent adversary and cranking alone isn't enough to
defeat an incoming missile, it's time for more drastic measures.
Dive to lower altitudes, where the thicker air creates more drag and makes
rocket motors less efficient.\punckern\footnote{A fun bit of rocket
science: rocket motors are most efficient in a vaccum.}
Fly a perpendicular heading to the missile---\introduce{beam} it.
Beaming maximizes the amount the missile has to turn,
and makes you harder to track on radar,
especially if you're below the one locking you up.
Dispense chaff or flares.
Finally, when the missile is close,\punckern\footnote{Knowing \emph{when} a
missile is seconds from impact comes with timing, experience, and a bit of luck.}
pull a high-G turn \emph{into} the missile and across its nose.
To cut the corner, the missile will have to turn even harder,
bleeding lots of energy.

All the same factors apply when you're the shooter.
The time to squeeze the trigger depends on:
\begin{itemize}
\item Range (obviuosly)
\item Altitude: higher is better, but being too high can make it hard to
    maneuver if you get shot at.
\item Aspect \& closure rate:
    A missile has less distance to cover if you and your target are racing
    towards each other than if the target is running away.
\end{itemize}
Once you've locked onto a target,
your jet's avionics will calculate a few ranges and display them on your
heads-up display \ac{(hud)}.
These usually include:
\begin{itemize}
\item The maximum aerodynamic range of the missile, usually called
    \fakesub{R}{aero}.
    A shot at this range will only hit if the target doesn't maneuver
    \emph{at all} after you launch.
\item The maximum range where a hit is \emph{probable},
    assuming the target continues on its current course at its current airspeed.
    This is usually called \fakesub{R}{max} or \fakesub{R}{pi}.
\item The maximum range where a hit is probable
    even if the target turns and runs away at its current airspeed
    as soon as you launch.
    This is usually called \fakesub{R}{tr} (Range, turn \& run)
    or \fakesub{R}{ne} (Range, no escape).
\item The minimum range where the missile has enough room to turn into the
    enemy and arm itself, called \fakesub{R}{min}.
\end{itemize}
Notice that all of these ranges are expressed as probabilities!
Even if the missile works perfectly,
you don't know how your target maneuver once you've shot at them.
A shot inside \fakesub{R}{ne} could still be defeated if the enemy can accelerate,
break your radar lock, or outmaneuver the missile as it gets close.
Holding your shot (above \fakesub{R}{min}) will almost always give you
a better chance at a kill,
but it also increases the likelyhood you'll get hit yourself.

Keeping all that in mind, let's talk tactics.

\section{The SARH fight}

Semi-active radar homing \ac{(sarh)} missiles like the \designation{AIM-7}
lack their own radar emitter, so to home in on your target,
you must continuously ``paint'' it with your radar in single-target track
\ac{(stt)} mode.
This presents two unique challenges:
\begin{enumerate}
\item Any maneuvering you do must keep the target inside the limits of your radar
    gimbals. In other words,
    you have to keep flying towards the target until your missile hits.
\item Since your radar is only pointing at the target in \ac{stt} mode,
    so you lose the \ac{sa} it would normally give you.
\end{enumerate}

\section{The ARH fight}

Active radar homing \ac{(arh)} missiles like the \designation{AIM-120} AMRAAM
make your job easier for one reason: you can turn around.

\chapter{Dogfighting}
