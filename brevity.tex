\chapter{Brevity}

Like most professions, combat aviation has its own jargon.
Called \emph{brevity}, it's incomprehensibly dense to a newcomer,
but it lets pilots communicate quickly in the heat of battle.

Most folks you find online will be using \textsc{us/nato} brevity.
There's many,
many terms,\punckern\footnote{An early-2000s version of an official
multiservice \textsc{us} training guide is declassified and freely available
online at \https{apps.dtic.mil/dtic/tr/fulltext/u2/a404426.pdf}}
but we'll cover some of the most common ones.
Before we dive in, know two things:
\begin{enumerate}
\item If you can't think of a term you want,
    \emph{use plain language!}
    The main goal is to be understood, quickly.
    Sounding cool is just an added bonus.
\item Some terms have one version for friendly forces
    and another for hostiles or unknown forces.
    Don't mix them up!
    If you're \emph{visual} on your wingman,
    you can fly in formation with them.
    But if you \emph{tally} your wingman,
    they'll wonder why you're trying to shoot them.
\end{enumerate}

\section{Combat}

Finding targets and shooting them.
\begin{description}
\item[Contact] You notice a potential target on your sensors (usually radar).
    Provide a location and additional information if you can.

    ``Contact, Bullseye 120 35. Hot.'' CHECKME

\item[Fox 1] Friendly launch of a semi-active radar homing \acronym{(sarh)}
    missile, like an \designation{AIM-7} Sparrow.
\item[Fox 2] Friendly launch of a heat-seeking missile, like an
    \designation{AIM-9} Sidewinder.
\item[Fox 3] Friendly launch of an active radar homing \acronym{(arh)} missile,
    like an \designation{AIM-120} AMRAAM.

\item[Paveway] Friendly is dropping laser-guided bombs,
    like the Paveway~II series (\designation{GBU-10/12/16})

\item[Pickle] Friendly is dropping bombs (usually unguided),
    like the Mark~80 series.

\item[Splash] Target destroyed. Sometimes \textbf{Shack} is used for
    air-to-ground impacts.

\item[Rifle] Friendly launch of an air-to-ground missile,
    like an \designation{AGM-65} Maverick.
\end{description}
